\font\manual=manfnt
\def\dbend{{\manual\char127}}
\def\warn#1{\begin{itemize}\item[\dbend]#1\end{itemize}}
\def\red#1{{\color{red}#1}}
\def\redtt{\color{red}\tt}
\def\bluett{\color{blue}\tt}
\font\manual=manfnt
\def\dbend{{\manual\char127}}
\def\eqq{{\buildrel?\over=}}
\def\I{{\cal I}}
\def\Z{{\bf Z}}
\def\Q{{\bf Q}}
\def\C{{\bf C}}
\def\N{{\bf N}}
\def\R{{\bf R}}
\def\CL{\mathop{\rm CL}}
\def\IDA{{}_{\rm DA}\int}
\def\DDA{D_{\rm DA}}
\def\arctanDA{\arctan_{\rm DA}}
\def\logDA{\log_{\rm DA}}
\def\cDA{constant${}_{\rm DA}$}
\def\fracDDA#1#2{\frac{{\rm d}#1}{{\rm d}_{\rm DA}#2}}
\def\Ied{{}_{\epsilon\delta}\int}
\def\Ded{D_{\epsilon\delta}}
\def\arctaned{\arctan_{\epsilon\delta}}
\def\loged{\log_{\epsilon\delta}}
\def\fracDed#1#2{\frac{{\rm d}#1}{{\rm d}_{\epsilon\delta}#2}}
\bibliographystyle{alpha}
%\documentclass[notes=hide]{beamer}   % Without notes
\documentclass[handout]{beamer}   % handout mode 
\usepackage[final]{pdfpages}

%\mode<presentation>

\usetheme{Warsaw}

\setbeamertemplate{navigation symbols}{}

\usepackage{verbatim}
\usepackage{url}
\usepackage{color}

\title[Reshaping \dots{} Bath]{REshaping Assessment Excellence: COVID-19 and the New Now at the University of Bath}
\author{James Davenport \& Tom Crick}
%(Thanks to RJB for the improved title)}
\institute{University of Bath}%(visiting Waterloo)}
\date{19 September 2022}%\\($\cal J$=JHD's notes; $\cal C$=Chapman:\\ \emph{MATLAB(R) Programming with Applications for Engineers})\\This lecture is being recorded for the benefit of late arrivals}
\expandafter\def\expandafter\insertshorttitle\expandafter{%
  \insertshorttitle\hfill%
  \insertframenumber\,/\,\inserttotalframenumber}

\begin{document}

\frame{
\titlepage
}
\begin{frame}[fragile]
\frametitle{Setting}
\pause
A common remark in commercial IT circles is that Covid did more for ``digital transformation'' in a month than cohorts of Chief  Information Officers etc. had accomplished in a decade. 
\pause\par
 Many businesses are not going back to ``five days a week'', with some research \cite{Bindley2022a} suggesting that ``two days is optimal''.
\par\pause
There are many changes in Higher Education as well, which are not going to be reversed: we look at one area: assessment, and in particular the written examination.
%\cite{Jensenetal2014a}
\end{frame}
\begin{frame}[fragile]
\frametitle{Assessment pre-Covid}
\pause
Indeed, the final examination is a university institution that would appear to be off-limits as far as innovation is concerned. To put this into context, while faculty and students alike will not stray too far from a computer as they go about their daily business, it is still the norm for examinations to be conducted using pen and paper. Does this imply, therefore, that some element of modern learning theory might be sacrificed if it were abandoned in favour of some alternative instrument? Or, given it is still the most commonly administered summative assessment instrument in universities today, is there some other special intrinsic value attached to a closed-book, invigilated exam that justifies its continued use?
\cite{WilliamsWong2009a}
\end{frame}
\begin{frame}[fragile]
\frametitle{Forms of Examination}
\pause
\begin{description}
\item[Trad-C]Invigilated in an ``examination hall''%, with no technology or other resources except the question paper.
\item[Trad-OR As above but students can bring in specified paper resources: often called “open book”. That specification could be very precise: “A clean copy of the fifth edition of …”, or as vague as “a binder of own notes”. In practice invigilators find this hard to enforce in a large examination., e.g. that a textbook has no annotations.
\item[Trad-OU As above with no restrictions on what can be brought it (on paper).
\item[TakeHome-OU In a take-home examination, the student is given the question paper, and has to bring the answer back later (generally 24-hours). Bengtsson (2019) is a useful survey of these, and has two key points. 
1.	It is concluded that take-home exams may be the preferred choice of assessment method on the higher taxonomy levels because they promote higher-order thinking skills and allow time for reflection.
2.	Due to the obvious risk of unethical student behaviour, take-home exams are not recommended on the lowest taxonomy level.
JHD had experimented with this in the past for CM50209 Cybersecurity.
\item[Interim Use a Virtual Learning Environment to deliver an examination paper, and collect answers. There are no technological constraints on the help students could acquire.
\item[Electronic-C A university-managed examination, generally using a specific software platform (Bath used Inspera, but the precise choice is probably irrelevant) for delivery and submission. The students are not allowed to use any other resources, but there are wide variations on how this is enforced, from an honour system, through restricted browsers  to full AI-based monitoring. Though it has been sold as a panacea, AI-based monitoring has its limitations (New York Times (Kashmir Hill), 2022) and may be illegal in some jurisdictions (JISC, 2022).
\item[Electronic-OU As above, but the students are allowed to use any Internet resources. The wording here is not standardised, but the intention is that the students can consult internet resources, but not people. This raises unsolved questions around “intelligent”’ resources, notably those that can write answers to programming exercises (Finnie-Ansley.J. et al., 2022).

\end{description}

\end{frame}
\begin{frame}[fragile]
\frametitle{}
\pause
\end{frame}
\begin{frame}[fragile]
\frametitle{}
\pause
\end{frame}
\iffalse
Corpus Reformatorum edidit Carolus Gottlief Bretschneider Volumen XI, 1843.
Nam primum si spectabimus utilitatem, quam affert cunsuetudo examinum, praesertim huic aetati, in cuius praecipue gratiam hi primi gradus instituti sunt, tantam eam comperiemus, ut nlla amplior nec uberior in schola possit esse. p. 182
For in the first place, if we look at the advantage which the custom of examinations brings, especially to this age, in whose favor these first grades are especially instituted, we shall find it to be so great that nothing can be greater or more abundant in the school.
Magna haec quidem sunnt, sed est et alia non minor utilitas examinum, quod cohercent ac cohibent cursum studiorum intra certos limites. p.185.
These are indeed great, but there is also another, not less, advantage of examinations, that they control and control the course of studies within certain limits.
Diximus de examinum utilitate et si ea tanta et tam multiplex est, ut nec in hac temporis angustia, neque a ne homine at dicendum minima instructo satis commemorari possit. [p. 187]
We have spoken of the usefulness of examinations, and if they are so great and so complex, that they cannot be sufficiently mentioned, neither in the straits of this time, nor by a man who is, to say the least, instructed.
\fi
\begin{frame}[fragile]
\frametitle{}
\pause
\begin{itemize}[<+->]
\item
\end{itemize}
\end{frame}
\begin{frame}[fragile]
\frametitle{}
\pause
\end{frame}

\begin{frame}[allowframebreaks]
\frametitle{Bibliography}
\bibliography{../../../jhd}
\end{frame}
\end{document}
\begin{frame}[fragile]
\frametitle{}
\pause
\begin{itemize}[<+->]
\end{itemize}
\end{frame}
\begin{frame}[fragile]
\frametitle{}
\pause
\begin{verbatim}
\end{verbatim}
\end{frame}
\begin{frame}[fragile]
\frametitle{}
\pause
\begin{itemize}[<+->]
\item 
\end{itemize}
\end{frame}
\pause
\begin{description}
\end{description}
\begin{verbatim}
\end{verbatim}
\begin{frame}[fragile]
\frametitle{}
\pause
\begin{itemize}[<+->]
\item 
\end{itemize}
\end{frame}
\begin{frame}[fragile]
\frametitle{}
\pause
\begin{itemize}[<+->]
\item 
\end{itemize}
\end{frame}
\begin{frame}[fragile]
\frametitle{}
\pause
\begin{itemize}[<+->]
\item 
\end{itemize}
\end{frame}
\begin{frame}[fragile]
\frametitle{}
\pause
\begin{itemize}[<+->]
\item 
\end{itemize}
\end{frame}
