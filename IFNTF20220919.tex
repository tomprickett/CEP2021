\font\manual=manfnt
\def\dbend{{\manual\char127}}
\def\warn#1{\begin{itemize}\item[\dbend]#1\end{itemize}}
\def\red#1{{\color{red}#1}}
\def\redtt{\color{red}\tt}
\def\bluett{\color{blue}\tt}
\font\manual=manfnt
\def\dbend{{\manual\char127}}
\def\eqq{{\buildrel?\over=}}
\def\I{{\cal I}}
\def\Z{{\bf Z}}
\def\Q{{\bf Q}}
\def\C{{\bf C}}
\def\N{{\bf N}}
\def\R{{\bf R}}
\def\CL{\mathop{\rm CL}}
\def\IDA{{}_{\rm DA}\int}
\def\DDA{D_{\rm DA}}
\def\arctanDA{\arctan_{\rm DA}}
\def\logDA{\log_{\rm DA}}
\def\cDA{constant${}_{\rm DA}$}
\def\fracDDA#1#2{\frac{{\rm d}#1}{{\rm d}_{\rm DA}#2}}
\def\Ied{{}_{\epsilon\delta}\int}
\def\Ded{D_{\epsilon\delta}}
\def\arctaned{\arctan_{\epsilon\delta}}
\def\loged{\log_{\epsilon\delta}}
\def\fracDed#1#2{\frac{{\rm d}#1}{{\rm d}_{\epsilon\delta}#2}}
\bibliographystyle{alpha}
%\documentclass[notes=hide]{beamer}   % Without notes
\documentclass[handout]{beamer}   % handout mode 
\usepackage[final]{pdfpages}

%\mode<presentation>

\usetheme{Warsaw}

\setbeamertemplate{navigation symbols}{}

\usepackage{verbatim}
\usepackage{url}
\usepackage{color}

\title[Reshaping \dots{} Bath]{REshaping Assessment Excellence: COVID-19 and the New Now at the University of Bath}
\author{James H. Davenport \& \hbox{Tom Crick\ \qquad\ \qquad}\\
\small\tt J.H.Davenport@bath.ac.uk \& Thomas.Crick@swansea.ac.uk}
%(Thanks to RJB for the improved title)}
\institute{University of Bath \& Swansea University}%(visiting Waterloo)}
\date{19 September 2022}%\\($\cal J$=JHD's notes; $\cal C$=Chapman:\\ \emph{MATLAB(R) Programming with Applications for Engineers})\\This lecture is being recorded for the benefit of late arrivals}
\expandafter\def\expandafter\insertshorttitle\expandafter{%
  \insertshorttitle\hfill%
  \insertframenumber\,/\,\inserttotalframenumber}

\begin{document}

\frame{
\titlepage
}
\begin{frame}[fragile]
\frametitle{Setting}
\pause
A common remark in commercial IT circles is that Covid did more for ``digital transformation'' in a month than cohorts of Chief  Information Officers etc. had accomplished in a decade. 
\pause\par
 Many businesses are not going back to ``five days a week'', with some research \cite{Bindley2022a} suggesting that ``two days is optimal''.
\par\pause
There are many changes in Higher Education as well, which are not going to be reversed: we look at one area: assessment, and in particular the written examination.
%\cite{Jensenetal2014a}
\end{frame}
\begin{frame}[fragile]
\frametitle{Assessment pre-Covid}
\pause
Indeed, the final examination is a university institution that would appear to be off-limits as far as innovation is concerned. To put this into context, while faculty and students alike will not stray too far from a computer as they go about their daily business, it is still the norm for examinations to be conducted using pen and paper. Does this imply, therefore, that some element of modern learning theory might be sacrificed if it were abandoned in favour of some alternative instrument? Or, given it is still the most commonly administered summative assessment instrument in universities today, is there some other special intrinsic value attached to a closed-book, invigilated exam that justifies its continued use?
\cite{WilliamsWong2009a}
\end{frame}
\begin{frame}[fragile]
\frametitle{Forms of Examination (Partial)}
\pause
\begin{description}[<+->]
\item[Trad-C]Invigilated in an ``examination hall''. JHD has used this.%, with no technology or other resources except the question paper.
\item[Trad-OR]As above but students can bring in specified paper resources: often called ``open book''. JHD has used this \cite{Betteridgeetal2019a} to reduce ``memorisation''.
%That specification could be very precise: “A clean copy of the fifth edition of …”, or as vague as “a binder of own notes”. In practice invigilators find this hard to enforce in a large examination., e.g. that a textbook has no annotations.
\item[Trad-OU] As above with no restrictions on what can be brought it (on paper).
\item[TakeHome-OU]In a take-home examination, the student is given the question paper, and has to bring the answer back later (generally 24-hours). JHD had used this.
%Bengtsson (2019) is a useful survey of these, and has two key points. 
%1.	It is concluded that take-home exams may be the preferred choice of assessment method on the higher taxonomy levels because they promote higher-order thinking skills and allow time for reflection.
%2.	Due to the obvious risk of unethical student behaviour, take-home exams are not recommended on the lowest taxonomy level.
%JHD had experimented with this in the past for CM50209 Cybersecurity.
\item[Interim]Use a VLE to deliver an examination paper, and collect answers. There are no technological constraints on the help students could acquire.
\item[Electronic-C]A university-managed examination, generally using specific software. How is ``Closed'' monitored?
% (Bath used Inspera, but the precise choice is probably irrelevant) for delivery and submission. The students are not allowed to use any other resources, but there are wide variations on how this is enforced, from an honour system, through restricted browsers  to full AI-based monitoring. Though it has been sold as a panacea, AI-based monitoring has its limitations (New York Times (Kashmir Hill), 2022) and may be illegal in some jurisdictions (JISC, 2022).
\item[Electronic-OU]As above, but the students are allowed to use any Internet resources. %The wording here is not standardised, but the intention is that the students can consult internet resources, but not people. This raises unsolved questions around “intelligent”’ resources, notably those that can write answers to programming exercises (Finnie-Ansley.J. et al., 2022).
\end{description}
\end{frame}
\begin{frame}[fragile]
\frametitle{The Bath timeline}
\pause
\begin{description}[<+->]
\item[Prior]Two hour Trad-C or Trad-O. TakeHome-OU disguised as ``coursework''.
\item[May 2020]{\bf Interim}. %Given the timing, Interim was the only practicable solution. Since many solutions were “at home” across the world, the examinations were still aimed at taking two hours, but 
All students were given a 24-hour window in which to do them. 
%Very few staff had experience in “open book” examinations, and certainly not when the whole Internet was an open book. Now we needed to have one examination per day, which stretched the examination period.
\item[January 2021]{\bf Electronic-O},
%We moved to Electronic-O. Because students were in different time zones, it was felt that a fixed start time was impossible, so there was 
still with a 24-hour window.  Maths took the option to insist that students only had three hours (2 hour exam+1/administration).
% (conceived of as a 2 hour exam plus an hour for administration) to complete the examination from starting the process. But the students could still choose their start time (based on their home time zone) as long as the exam was done in the 24-hour window.
\item[May 2021] 
%Based on the success of the Mathematical Sciences limited time experiment, and probably because staff now had more experience of setting open-book exams and getting the time requirements roughly right , 
Success, and greater familiarity with Electronic-O, meant
many more departments moved to the three-hour limit.
\item[January 2022] 
%Now that students were “largely expected” to be at Bath, the University kept to 
three-hour (still thought of as 2+1) examinations for all, but now fixed the start time.
%, rather than allowing a 24-hour window.
\item[May 2022] Following very substantial pressure by the academics, the university allowed some {\bf Trad-C} examinations in first-year subjects.
\end{description}
\end{frame}
\begin{frame}[fragile]
\frametitle{The student verdict (at Bath)}
This has been overwhelmingly positive among final year students (as captured by National Student Survey). One of many (22 in Maths, 13 in CS)
\begin{quote}
Online exams - the exams work better when they use problems to check your understanding instead of your memory.
\end{quote}
\pause
The students show appreciation of the different things that open-book examinations test,though they assume ``online''=``open book'' --- it was for them. \pause
The rare negative ones are worth noting.
\pause
\begin{quote}
I absolutely would not want to keep online exams, in person is much better due to an abundance of cheating.
\end{quote}
\pause
\begin{quote}
Certain exams are more relevant in open-book, online format, but not all.
\end{quote}
\end{frame}
\begin{frame}[fragile]
\frametitle{Academic Misconduct}
\cite{Dickinson2022a} reports a small (N=900) survey \cite{AcademicAppeals2022a}  of UK students. 
\pause
\begin{quote}
The numbers suggest that 1 in 6 students in the UK have cheated in online exams this academic year. Over half of those surveyed knew people who had cheated in online assessments. Almost 8 out 10 believed that it was easier to cheat in online exams than in exam halls, and the methods for cheating were often laughably rudimentary – including calling or messaging friends for help during the exam, using google to search for answers on a separate device, or asking parents to read through answers prior to submission.
\end{quote}
\pause
The University of Bath has certainly experienced an increase in detected use of ``unfair means''. There is probably a larger increase in undetected use of unfair means.
\pause\par
\end{frame}
\iffalse
Corpus Reformatorum edidit Carolus Gottlieb Bretschneider Volumen XI, 1843.
Nam primum si spectabimus utilitatem, quam affert consuetudo examinum, praesertim huic aetati, in cuius praecipue gratiam hi primi gradus instituti sunt, tantam eam comperiemus, ut nulla amplior nec uberior in schola possit esse. p. 182
For in the first place, if we look at the advantage which the custom of examinations brings, especially to this age, for whose favor these first grades are especially instituted, we shall find it to be so great that nothing can be greater or more abundant in the school.
Magna haec quidem sunnt, sed est et alia non minor utilitas examinum, quod cohercent ac cohibent cursum studiorum intra certos limites. p.185.
These are indeed great, but there is also another, not less, advantage of examinations, that they control and control the course of studies within certain limits.
Diximus de examinum utilitate et si ea tanta et tam multiplex est, ut nec in hac temporis angustia, neque a ne homine at dicendum minima instructo satis commemorari possit. [p. 187]
We have spoken of the usefulness of examinations, and if they are so great and so complex, that they cannot be sufficiently mentioned, neither in the straits of this time, nor by a man who is, to say the least, instructed.
\fi
\begin{frame}[fragile]
\frametitle{Academic Misconduct: JHD's experiences}
\pause
\begin{itemize}[<+->]
\item JHD has sat on misconduct judgement panels, and his subjective view would be that much of this has been extempore abuse, as students get tempted in the stress of the examination, rather than pre-planned (as cheating in Trad-C examinations has to be).
\item One of JHD's final year examination papers (in the 24-hour setting) was posted on Chegg. No responses were detected.
\item[\dbend]But it might be very different for first-year examinations, and Maths (staff and students) really wanted {\bf Trad-C} here.
\item ``I have an online Statistics Examination from 08:30 to 11:30 next Friday. Is anyone familar with ``Definition of continuous random variables (RVs), cumulative distribution functions (CDFs) and probability density functions (PDFs).
Some common continuous distributions including uniform, exponential and normal. \dots'' [a screenshot of what was obviously a syllabus] please bid to help me. [\url{https://www.freelancer.com/jobs/statistics/}].
\end{itemize}
\end{frame}
\begin{frame}[fragile]
\frametitle{Challenges of Online Exams}
\pause
\begin{itemize}[<+->]
\item{\bf Setting}. Open-book exams are very different, and it takes practice to get the difficulty and duration right. The entire community (examiners, mentors, externals) have had to adjust.
\item{\bf Misconduct} --- note that ``AI proctoring'' has problems, from legal \cite{JISC2022a} to practical \cite{NewYorkTimes2022l}.  ``Automation bias'' training is probably essential.
\item{\bf Authenticity} --- is that the student the person we think? Some Italian universities, even pre-Covid, insisted on (short) oral examinations for the teacher to confirm identity and some knowledge.  Expensive, though.
\item{\bf Stress} on students' ethics --- do we prepare them for this? We have seen nothing on this.
\item{\bf Definition} of proper/improper online resources. We haven't seen a good definition that will allow Wikipedia searches but not tools that write programs \cite{FinnieAnsleyetal2022a}, say.
%\item{\bf }
%\item{\bf }
\end{itemize}
\end{frame}

\begin{frame}[allowframebreaks]
\frametitle{Bibliography}
\bibliography{../../../jhd}
\end{frame}
\end{document}
\begin{frame}[fragile]
\frametitle{}
\pause
\begin{itemize}[<+->]
\end{itemize}
\end{frame}
\begin{frame}[fragile]
\frametitle{}
\pause
\begin{verbatim}
\end{verbatim}
\end{frame}
\begin{frame}[fragile]
\frametitle{}
\pause
\begin{itemize}[<+->]
\item 
\end{itemize}
\end{frame}
\pause
\begin{description}
\end{description}
\begin{verbatim}
\end{verbatim}
\begin{frame}[fragile]
\frametitle{}
\pause
\begin{itemize}[<+->]
\item 
\end{itemize}
\end{frame}
\begin{frame}[fragile]
\frametitle{}
\pause
\begin{itemize}[<+->]
\item 
\end{itemize}
\end{frame}
\begin{frame}[fragile]
\frametitle{}
\pause
\begin{itemize}[<+->]
\item 
\end{itemize}
\end{frame}
\begin{frame}[fragile]
\frametitle{}
\pause
\begin{itemize}[<+->]
\item 
\end{itemize}
\end{frame}
