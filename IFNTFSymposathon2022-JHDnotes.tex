\font\manual=manfnt
\def\dbend{{\manual\char127}}
\def\eqq{{\buildrel?\over=}}
\def\C{\mathbf{ C}}
\def\N{\mathbf{ N}}
\def\Q{\mathbf{ Q}}
\def\R{\mathbf{ R}}
\def\Z{\mathbf{Z}}
\def\F{\mathbf{F}}
\def\softO{\tilde{\cal O}}
\def\O{{\cal O}}
%\def\N{{\bf N}}
%\def\Q{{\bf Q}}
%\def\R{{\bf R}}
%\def\Z{{\bf Z}}
\def\action#1{\hfil\rlap{\bf#1}\break}
\def\pcite#1{[#1]}
\def\RootOf{\mathop{\rm RootOf}\nolimits}
\def\r#1#2{``#1''$\rightarrow$``#2''}
\documentclass{article}
\bibliographystyle{alphaurl}
\usepackage[hyphens]{url}
%\usepackage{enumitem}
\usepackage{pdfpages}
\usepackage{verbatim}
\usepackage{hyperref}
\usepackage{combelow}
\usepackage[show]{ed}
\usepackage{graphicx}
\newtheorem{proposition}{Proposition}
\newtheorem{theorem}{Theorem}
\newtheorem{lemma}{Lemma}
\newtheorem{corollary}{Corollary}
\newtheorem{definition}{Definition}
\newtheorem{notation}{Notation}
\newtheorem{example}{Example}
\newtheorem{problem}{Problem}
\def\decision#1{\par{\bf #1}}
\def\action#1{\hfil\rlap{\bf#1}\break}
%\url{https://assessment.bath.ac.uk/admin#grading/overview/78727640}
\pagestyle{empty}
\begin{document}
\author{Notes by J.H.Davenport}
\title{}
\date{19 September 2022}
\maketitle
\def\r{$\rightarrow$}
\section{Practice what you Preach: Laura Ritchie}
Pedagogical change, like many others, takes thrity years or more to be aadopted, but adoption has sped up. The pandemic has greatly sped things up even more.
\par
Note that I have watched the Queen's funeral, or films, ot TikTok, as purely passive events. teaching/Learning can be same. But the pandemic has given is many opportunities to change.
\par
But we need to take care of own personal development. What is your new normal?  ``Put your own mask on first before helping others''.
\section{JHD}
\begin{description}
\item[Q]Thank you for those insights. For me, the main problem with Open Book exams is that staff write closed-book questions for them, and then complain that the answers they are getting are just "my lecture notes parroted back at me!"  Convincing them that they are the cause of the problem is... err... 'challenging'
\item[A]@steve and all: yes - we had some good messages (partly with my help) about the difference and the need to rethink.  If anything, some staff went the other way, and the weaker students really struggled.
\item[Q]To what extent have your colleagues who deal specifically with academic misconduct had to adapt their approaches
\item[A]We needed to appoint additional assessors for examination misconduct  cases.
\item[Q--Clive]For some purposes and situations (smaller cohorts?), oral exams surely have a role to play again.
\item[A]Some Italian universities have always had oral examinations as well: basically to detect misconduct and possibly flag special cases.  Quite often 5-10 minutes long.
\item[A-Steve]@Clive:  What I find interesting is that oral exams are seen by many as inappropriate, but are the fundamental way in which PhDs are examined - supposedly the highest and most rigorous level of academic achievement
\item[Steve]Is the problem with academic misconduct more that we design assessments that can be cheated, rather than designing assessments that are cheat-proof (or at least make it much harder to cheat in). I wonder what effect it would have if we spent as much effort into that, as we have (as a sector) into proctoring mechanisms.
\item[A--JHD]@steve/all: hard to make an exam which is "cheat-proof" iterms of hiring a professional to help you take the assessment.
\item[Steve]@James: By Cheat-proof, I mean in the *design* of the assessment, rather than the way in which we deliver the assessment.
\item[Clive]I had a small  MBA elective which was entirely experiential relating to professional expertise and insight. I could only come up with oral as a way to probe what students had learned. They were VERY apprehensive, but was quite effective in grading. External required they were audio recorded.
\end{description}
\section{The lived experience: Jess Power (Salford)}
Pace of change much greater, but new opportunities. Redesigning our strategy for the next five years.
\par
Had started a new ``enabling student success'' project in Spring 2020 just before Covid. Why 
\section{Alighing HE Educator/Learner proactices with Post-digital: Nic Fair (Southampton)}
BRIDGES research project: 403 staff, 900+ PG(R). Pre-Covid, 50\% of teachers did no online, during 78\%, now a bell-curve with median at about 50\%.
\par
Forum/Blog/MOOC acually rarely used.
\par
Claims that 79\% of dialy interactions are digitally mediated. 98\% of searchng is done digitally; 94\% of information gathering, 95\% of communicatig and collaboration. More than 54\% of student interactions are with non-humans (websites etc.). But only 2\% with powerpoint/word.
\begin{description}
\item[Q]
\item[A]
\end{description}
\section{}
\begin{description}
\item[Q]
\item[A]
\end{description}
\section{}

\section{}
\begin{description}
\item[Q]
\item[A]
\end{description}
\begin{description}
\item[]
\end{description}
\bibliography{../../../jhd}
\end{document}
\section{}
\begin{description}
\item[Q]
\item[A]
\end{description}
\begin{enumerate}
\item 
\end{enumerate}
\begin{quote}
\end{quote}
\begin{itemize}
\item
\end{itemize}
\begin{example}
\end{example}
\begin{definition}
\end{definition}
\begin{theorem}
\end{theorem}
\begin{description}
\item[Theme 1]
\item[Theme 2]
\item[Theme 3]
\item[Theme 5]
\item[Theme 6]
\item[Monitoring]
\end{description}
\begin{enumerate}
\item 
\end{enumerate}
\begin{quote}
\end{quote}
