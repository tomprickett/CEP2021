%%%% anonymous-acm.sty  
%% version 1.0 09-May-2020   
%% Maintained by Brett A. Becker: brett.becker@ucd.ie

% This work may be distributed and/or modified under the
% conditions of the LaTeX Project Public License, either 
% version 1.3 of this license or any later version.
% The latest version of this license is in
% http://www.latex-project.org/lppl.txt
% and version 1.3 or later is part of all distributions of LaTeX
% version 2005/12/01 or later.

% This program is distributed in the hope that it will be useful 
% but WITHOUT ANY WARRANTY; without even the implied warranty of 
% MERCHANTABILITY or FITNESS FOR A PARTICULAR PURPOSE.

% designed to be used with acmart.cls - required for full functionality

%% Identification
%% The package identifies itself and the LaTeX version needed
\ProvidesPackage{anonymous-acm}
\NeedsTeXFormat{LaTeX2e}


% Define anonymous condition
\newif\ifAnonCondition\AnonConditiontrue

% Declare options
\DeclareOption{true}{\AnonConditiontrue}
\DeclareOption{false}{\AnonConditionfalse}
\DeclareOption*{\PackageWarning{anonymous-acm}{Unknown ‘\CurrentOption’}}
\ProcessOptions\relax

% Anonymous Authors
\newcommand{\authoranon}[1]{#1}
\ifAnonCondition
\renewcommand{\authoranon}[1]{
\author{Anonymous Author(s)}
}
\else
\renewcommand{\authoranon}[1]{#1}
\fi

% Anonymous Arbitrary Text
\newcommand{\textanon}[2]{#1}
\ifAnonCondition
\renewcommand{\textanon}[2]{\ifstrequal{}{#2}{<text removed for peer review>}{#2}}
\else
\renewcommand{\textanon}[2]{#1}
\fi

% Anonymous Links
\newcommand{\linkanon}[2]{\href{#1}{#2}}
\ifAnonCondition
\renewcommand{\linkanon}[2]{<anonymous link>}
\else
\renewcommand{\linkanon}[2]{\href{#1}{#2}}
\fi

\newcommand{\textlinkanon}[2]{\href{#1}{#2}}
\ifAnonCondition
\renewcommand{\textlinkanon}[2]{#2}
\else
\renewcommand{\textlinkanon}[2]{\href{#1}{#2}}
\fi

% Anonymous Citations & References
\newcommand{\citeanon}[2][\ ]{\cite[#1]{#2}}
\ifAnonCondition
\renewcommand{\citeanon}[2][\ ]{[anonymous]}
\else
\renewcommand{\citeanon}[2][@]{\ifstrequal{@}{#1}{\cite{#2}}{\cite[#1]{#2}}}
\fi

% Anonymous Acknowledgments
\ifAnonCondition
\excludecomment{acks}
\fi

%%
%% This is file `sample-sigconf.tex',
%% generated with the docstrip utility.
%%
%% The original source files were:

%%
%% samples.dtx  (with options: `sigconf')
%% 
%% IMPORTANT NOTICE:
%% 
%% For the copyright see the source file.
%% 
%% Any modified versions of this file must be renamed
%% with new filenames distinct from sample-sigconf.tex.
%% 
%% For distribution of the original source see the terms
%% for copying and modification in the file samples.dtx.
%% 
%% This generated file may be distributed as long as the
%% original source files, as listed above, are part of the
%% same distribution. (The sources need not necessarily be
%% in the same archive or directory.)
%%
%% The first command in your LaTeX source must be the \documentclass command.
\documentclass[sigconf]{acmart}
\usepackage[true]{anonymous-acm}


%%%% As of March 2017, [siggraph] is no longer used. Please use sigconf (above) for SIGGRAPH conferences.

%%%% As of May 2020, [sigchi] and [sigchi-a] are no longer used. Please use sigconf (above) for SIGCHI conferences.

%%%% Proceedings format for SIGPLAN conferences 
% \documentclass[sigplan, anonymous, review]{acmart}

%%%% Proceedings format for conferences using one-column small layout
% \documentclass[acmsmall,review]{acmart}

%%
%% \BibTeX command to typeset BibTeX logo in the docs
\AtBeginDocument{%
  \providecommand\BibTeX{{%
    \normalfont B\kern-0.5em{\scshape i\kern-0.25em b}\kern-0.8em\TeX}}}

%% Rights management information.  This information is sent to you
%% when you complete the rights form.  These commands have SAMPLE
%% values in them; it is your responsibility as an author to replace
%% the commands and values with those provided to you when you
%% complete the rights form.
\setcopyright{acmcopyright}
\copyrightyear{2021}
\acmYear{2021}
\acmDOI{10.1145/1122445.1122456}


%% These commands are for a PROCEEDINGS abstract or paper.
\acmConference[CEP '21]{Durham'21: ACM Computing Education Practice Conference}{January 07, 2021}{Durham, United Kingdom}
\acmBooktitle{CEP '21: ACM Computing Education Practice Conference,
  {January 07, 2021}{Durham, United Kingdom}
}
\acmPrice{15.00}
\acmISBN{978-1-4503-XXXX-X/18/06}


%%
%% Submission ID.
%% Use this when submitting an article to a sponsored event. You'll
%% receive a unique submission ID from the organizers
%% of the event, and this ID should be used as the parameter to this command.
%%\acmSubmissionID{123-A56-BU3}

%%
%% The majority of ACM publications use numbered citations and
%% references.  The command \citestyle{authoryear} switches to the
%% "author year" style.
%%
%% If you are preparing content for an event
%% sponsored by ACM SIGGRAPH, you must use the "author year" style of
%% citations and references.
%% Uncommenting
%% the next command will enable that style.
%%\citestyle{acmauthoryear}

%%
%% end of the preamble, start of the body of the document source.
\begin{document}

%%
%% The "title" command has an optional parameter,
%% allowing the author to define a "short title" to be used in page headers.
\title{Integrating New Lecturers into the UK Computer Science Community}

%%
%% The "author" command and its associated commands are used to define
%% the authors and their affiliations.
%% Of note is the shared affiliation of the first two authors, and the
%% "authornote" and "authornotemark" commands
%% used to denote shared contribution to the research.
\authoranon{
 \author{Alan Hayes}
 \authornote{All Authors contributed equally}
 \author{James H. Davenport}
 \authornotemark[1]
  \affiliation{%
   \institution{ University of Bath}
   \city{Bath}
   \country{UK}
 }
 \email{a.hayes@bath.ac.uk}
 \email{j.h.davenport@bath.ac.uk}


 \author{Tom Crick}
 \authornotemark[1]
 \affiliation{
   \institution{Swansea University}
   \city{Swansea}
   \country{UK}
 }
 \email{thomas.crick@swansea.ac.uk}


 \author{Alastair Irons}
 \authornotemark[1]
 \affiliation{
   \institution{ Sunderland University}
   \city{Sunderland}
   \country{UK} }
 \email{alastair.irons@sunderland.ac.uk}
 

 \author{Tom Prickett}
 \authornotemark[1]
 \affiliation{
   \institution{ Northumbria University}
   \city{Newcastle upon Tyne}
   \country{UK}
 }
 \email{tom.prickett@northumbria.ac.uk}

}

%%
%% By default, the full list of authors will be used in the page
%% headers. Often, this list is too long, and will overlap
%% other information printed in the page headers. This command allows
%% the author to define a more concise list
%% of authors' names for this purpose.
\renewcommand{\shortauthors}{Trovato and Tobin, et al.}

%%
%% The abstract is a short summary of the work to be presented in the
%% article.
\begin{abstract}
  To Do
  
  




\end{abstract}

%%
%% The code below is generated by the tool at http://dl.acm.org/ccs.cfm.
%% Please copy and paste the code instead of the example below.
%%
\begin{CCSXML}
<ccs2012>
<concept>
<concept_id>10003456.10003457.10003527</concept_id>
<concept_desc>Social and professional topics~Computing education</concept_desc>
<concept_significance>500</concept_significance>
</concept>
</ccs2012>
\end{CCSXML}

\ccsdesc[500]{Social and professional topics~Computing education}

%%
%% Keywords. The author(s) should pick words that accurately describe
%% the work being presented. Separate the keywords with commas.
\keywords{computer science education, community, induction}


%%
%% This command processes the author and affiliation and title
%% information and builds the first part of the formatted document.
\maketitle


\section{What is it?}	
%%A short description of the practice you're presenting
A short description of the practice you're presenting

This paper presents a progress report from a project exploring, developing and delivering a trial initiative to supplement the professional development opportunities that are available to Faculty Members new to teaching Computer Science within a single Higher Education with National activities in the United Kingdom. It describes the work undertaken to date to research, design and develop a training course for those new to teaching Computer Science in the United Kingdom. 

This initiative begin via discussions regarding how the computer science education community could be better supported at \textanon{BCS Academy of Computing Board}{BLINDED} \citeanon{BCSAcademy}. This was followed by the recruitment of a steering group of academics. This is formed of academics from a wide range of Universities. The experience of the academics involved is varied. It includes junior academics to more senior academics as well as members of the professoriate. The group includes several academics who have been recognised as National Teaching Fellows by Advanced HE \cite{AdvancedHENTF}. The views of this steering committee were combined with the outcomes of a workshop at the \textanon{ACM UK and Ireland Computing Education Research Conference 2021}{BLINDED Location} \citeanon{ACMUKICER}, to formulate a pilot training course. The training course will be first run in December 2020, so initial progress can be shared at this conference. This paper presents the outcomes from the workshop and an overview of the design of the trial run of the training programme.

\section{Why are you doing it?}
%%What happened before? What is it changing / replacing / improving? What gap is it filling?
What happened before? What is it changing / replacing / improving? What gap is it filling?


There are a number of discipline level challenges well documented within Computer Science Education. Failure rates can be high, learning programming for the first time remains a significant hurdle~\cite{davenport-et-al:latice2016,murphy-et-al:programming2017,simon-et-al:sigcse2018}, with a range of issues impacting failure rates~\cite{Watson:2014:FRI:2591708.2591749}. Student satisfaction as measured by satisfaction surveys is reported as commonly below that of other disciplines \cite{Sinclair2015}. Challenges related to the employment prospects of graduates have been reported as inferior to other disciplines\cite{shadbolt2016shadbolt}. Effectively addressing these pedagogic challenges involves a number of sophisticated and evolving pedagogic approaches. To be an adept Computer Science academic requires maintaining currency in terms of specialist subject knowledge but also fluency in how to respond to the pedagogic challenges the discipline presents.

The complex and contested demands of learning and teaching in UK\&I higher education make the early career of an academic challenging ~\cite{Thomas2015} and potentially lonely ~\cite{Foote2009}, especially when balanced against their research aspirations, and wider professional service commitments. Learning and teaching development in the UK commonly involves working towards Fellowship of the Higher Education Academy~\cite{fellowship} (now known as AdvanceHE), either by direct application or by an accredited university postgraduate course. Typically, this is supported by mentoring from within a department. The quality of learning provided will be promoted in part by the strength of the community of practice operating within the department~\cite{Bolander2008} and the communities of practice that exist at a national and international level~\cite{Thomas2015}.

Together, the challenges of the teaching the discipline and the challenges of learning to become an effective academic can be fraught. However mechanisms do exist to support this transition. For a number of years  mentoring has been recognised as a important mechanism in the development of academics. Clearly mentoring can be formal and informal. Formal mentoring problems are now common as part of the process of on boarding new Faculty members.  Commonly this takes the form of a senior academic mentoring a new Faculty member. Mentoring has been highlighted as part of a set of activities help diversify a Faculty \cite{Golubchik2018}.While such mechanisms can be very effective, limiting guidance and support to one source rather than a community can be limiting \cite{deJanasz}. This suggests that mentoring across the community could be of benefit.  Informal mentoring can also be very effective but it is not without issue. For example it has been reported that women can feel marginalised from informal mentoring partnerships \cite{gibson2006mentoring}.  Given the diversity of New Faculty in the United Kingdom this suggests adopting a more formal approach to cross institute mentoring may be beneficial. 

?TODO:More related to BAME?

This paper summaries an initiative to date to explore the potential to supplement the existing initiatives with further national level training and development opportunities.


\section{Where does it fit?}
%%A short description of your teaching context. You may, for instance, include a description of intake, class size, curriculum sequence; anything that's necessary for others to understand your situation. How do things work at your institution?

A short description of your teaching context. You may, for instance, include a description of intake, class size, curriculum sequence; anything that's necessary for others to understand your situation. How do things work at your institution?

In the UK, New permanent Faculty members have had varied prior experience including:
\begin{itemize}
\item Post Docs taking up their first academic appointment at the start of their academic career.
\item Postdocs who are helping out with teaching as part of their career development in preparation for applying for an academic position.
\item PhD students, aspiring to an academic career, and with significant teaching responsibilities.
\item Established Academics (international) new to the UK HE sector.
\item Junior academics who have been in their temporary post for a few years.
\item Those transitioning from FE or College into their first HE job.
\item Those transitioning from industry into their first HE job.
\item CS lecturers at HEIs who don’t currently specify that their lecturing staff undertake  a Postgraduate Qualification
\end{itemize}

In any given year, the number of individuals in this position will vary. However in most intuitions, in any given year there is usually at-least one faculty member in this position and there are over one hundred Higher Education Intuitions in the UK that teaching computing. As such there is not an insignificant number of individuals who could benefit from the training.

For the pilot training course will be run for New Faculty members based at the Universities of those who form the steering committee. The Pilot will run in December 20. The training programme, follows a blended approach consisting of live face-to-face sessions supported with the availability of the material on a supporting website. The programming includes live virtual sessions presented by colleagues from Computer Science Departments across the HEI community. The sessions will be practical in focus and include sharing of practice highlight, recognised good practice and sharing of mistakes made / pitfalls to avoid.


\section{Does it work?}	
%%How do you know? Give some evidence of effectiveness in context.
How do you know? Give some evidence of effectiveness in context.

The workshop  at the \textanon{ACM UK and Ireland Computing Education Research Conference 2021}{BLINDED Location} \citeanon{ACMUKICER}, focused upon addressing three research questions:
\begin{enumerate}
\item What current and future opportunities are there to engage new CS Faculty with the UK and Ireland Computer Science Education Community of Practice?
\item What is the potential to supplement institutional academic/research development opportunities with national development opportunities?
\item How can we continue to raise the profile and value of new research faculty members engaging with Computer Science Education Research?
\end{enumerate}
The input from the 21 attendees generated a number of fascinating insights into challenges new Faculty Members face. Key Themes that emerged include:

\subsection{Island of excellence}
There are a number of intuitions within the UK with significant research interests in computer science education (for example: \textanon{Glasgow, Kent, Lancaster Napier}{BLAH,BLAH,BLAH}). Other institutions lacked such support. It was suggested that in some cases further collaboration with Education departments may also be of benefit. Elsewhere, some colleagues reported the absence of Computer Science Education Research Groups and the related infrastructure to promote the importance of computer science education
\subsection{Competing priorities / Resource constraints}
Balancing alternative pressures came out strongly as a key challenge in raising interest in computer science education. Hence any training course has to be designed with direct benefit for attendees. 
\subsection{Variable University level support}
Variable university level support was identified. Some colleagues reporting little support New Faculty with respect to learning and teaching. Other reporting postgraduate training programmes or even Higher Apprenticeships.
\subsection{Local and other initiatives}
A number of local initiatives exist, together with a growing number of national initiatives (as discussed in the next section.) This suggests the appetite for initiatives such as this one.
\subsection{The community}
The national Computer Science Education Community was seen as a very positive development and further initiatives that intended to strengthen or grow the community appeared to be well received.
\subsection{Marketing / perception}
A few participants commented less favourable regarding how Computer Science Education was seen in their home universities, together with a wish to support initiatives the raised its profile and perceived value. 

\section{Who else has done this?}
%%Where did you get the idea from? (If from published reports, please include references). How did you find out about it? Was it easy/hard to adopt? What did you change?
Where did you get the idea from? (If from published reports, please include references). How did you find out about it? Was it easy/hard to adopt? What did you change?

It has been reported the the most commonly employed mechanism for continuing professional development (CPD) is conversations between peers \cite{King2004} . A number of communities exist to support these conversations at the local, national and international level.  Internationally this development is supported by the groupings such as ACM Special Interest Group in Computer Science Education (ACM SIGCSE) \cite{SIGCSE} or the IEEE Education Society \cite{IEEEES}.  These organisations run a number of conferences to promote discussion and dialog as well as dissemination of ideas. In the UK and Ireland a chapter of SIGCSE \cite{UKI-SIGCSE} further promotes these discussions by running a computer education practice conference (CEP) conference \cite{CEP} and a computing education research conference \cite{UKICER}. As an addition to the CEP conference is a workshop designed to help welcome new people to the community. Conversations are also promoted by the Raspberry PI Foundation via it Symposium \cite{PI_SYM} and online seminars \cite{PI_Sem}. However whilst new Faculty are welcome to attend the conferences and seminars are directly aimed a the needs of New Faculty members who may in any case find it challenging to find funding for attendance. The ACM SIGCSE chapter also holds twitter meets and runs has a journal club. Whilst New Faculty are welcome to join in these discussions, they are not really aimed at their needs.  Some universities have centres which further promote these conversations for example Edinburgh Napier Centre for Computer Education Research \cite{Napier}. However far from all Universities in the United Kingdom have dedicated centres such as this. In summary the International and National Initiatives help promote a community that experienced academics can access to provide CPD, however the CPD for New Faculty is less developed particularly for those at Universities who do not have CSE research groups or centres.

In addition to training programmes run by individual Universities, in the UK, the Higher Education Academy (HEA) runs training programmes for academics at different stages of their career \cite{HEATraining}. However this training is not discipline specific. The Council of Professors and Heads of Computing (CPHC) run occasional workshops in a variety of issues. The most pertinent offering for new Faculty has been the Chair in 10 Years workshop which is aimed at facilitating career planning. Whilst this is a well received contribution, it does not directly support the development needs of new Faculty with respect of pedagogy.

As well as generic pedagogic challenges, most disciplines including computer science have a range of discipline specific challenges. The Institute of Mathematics and its Applications (IMA) for a number of years has ran a training course intended to welcome new faculty to the community and help address key discipline related pedagogic challenges \cite{IMA}.




\section{What will you do next?}
%%Will you vary this, or develop it further?
Will you vary this, or develop it further?
\section{Why are you telling us this?}	
%%What is interesting, or useful, about this to someone else?
What is interesting, or useful, about this to someone else?



%% The acknowledgments section is defined using the "acks" environment
%% (and NOT an unnumbered section). This ensures the proper
%% identification of the section in the article metadata, and the
%% consistent spelling of the heading.
\begin{acks}
 
  All the authors' institutions are members of the Institute of Coding (IoC), an initiative funded by the Office for Students (England) and the Higher Education Funding Council for Wales. The BCS Academy of Computing and the IoC has supported the development of this paper
 

\end{acks}

%%
%% The next two lines define the bibliography style to be used, and
%% the bibliography file.
\bibliographystyle{ACM-Reference-Format}
\bibliography{NewLecturers}

%%
%% If your work has an appendix, this is the place to put it.
%%\appendix

\balance
\end{document}
\endinput
%%
%% End of file `sample-sigconf.tex'.
