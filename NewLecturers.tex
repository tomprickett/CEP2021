%%%% anonymous-acm.sty  
%% version 1.0 09-May-2020   
%% Maintained by Brett A. Becker: brett.becker@ucd.ie

% This work may be distributed and/or modified under the
% conditions of the LaTeX Project Public License, either 
% version 1.3 of this license or any later version.
% The latest version of this license is in
% http://www.latex-project.org/lppl.txt
% and version 1.3 or later is part of all distributions of LaTeX
% version 2005/12/01 or later.

% This program is distributed in the hope that it will be useful 
% but WITHOUT ANY WARRANTY; without even the implied warranty of 
% MERCHANTABILITY or FITNESS FOR A PARTICULAR PURPOSE.

% designed to be used with acmart.cls - required for full functionality

%% Identification
%% The package identifies itself and the LaTeX version needed
\ProvidesPackage{anonymous-acm}
\NeedsTeXFormat{LaTeX2e}


% Define anonymous condition
\newif\ifAnonCondition\AnonConditiontrue

% Declare options
\DeclareOption{true}{\AnonConditiontrue}
\DeclareOption{false}{\AnonConditionfalse}
\DeclareOption*{\PackageWarning{anonymous-acm}{Unknown ‘\CurrentOption’}}
\ProcessOptions\relax

% Anonymous Authors
\newcommand{\authoranon}[1]{#1}
\ifAnonCondition
\renewcommand{\authoranon}[1]{
\author{Anonymous Author(s)}
}
\else
\renewcommand{\authoranon}[1]{#1}
\fi

% Anonymous Arbitrary Text
\newcommand{\textanon}[2]{#1}
\ifAnonCondition
\renewcommand{\textanon}[2]{\ifstrequal{}{#2}{<text removed for peer review>}{#2}}
\else
\renewcommand{\textanon}[2]{#1}
\fi

% Anonymous Links
\newcommand{\linkanon}[2]{\href{#1}{#2}}
\ifAnonCondition
\renewcommand{\linkanon}[2]{<anonymous link>}
\else
\renewcommand{\linkanon}[2]{\href{#1}{#2}}
\fi

\newcommand{\textlinkanon}[2]{\href{#1}{#2}}
\ifAnonCondition
\renewcommand{\textlinkanon}[2]{#2}
\else
\renewcommand{\textlinkanon}[2]{\href{#1}{#2}}
\fi

% Anonymous Citations & References
\newcommand{\citeanon}[2][\ ]{\cite[#1]{#2}}
\ifAnonCondition
\renewcommand{\citeanon}[2][\ ]{[anonymous]}
\else
\renewcommand{\citeanon}[2][@]{\ifstrequal{@}{#1}{\cite{#2}}{\cite[#1]{#2}}}
\fi

% Anonymous Acknowledgments
\ifAnonCondition
\excludecomment{acks}
\fi

%%
%% This is file `sample-sigconf.tex',
%% generated with the docstrip utility.
%%
%% The original source files were:

%%
%% samples.dtx  (with options: `sigconf')
%% 
%% IMPORTANT NOTICE:
%% 
%% For the copyright see the source file.
%% 
%% Any modified versions of this file must be renamed
%% with new filenames distinct from sample-sigconf.tex.
%% 
%% For distribution of the original source see the terms
%% for copying and modification in the file samples.dtx.
%% 
%% This generated file may be distributed as long as the
%% original source files, as listed above, are part of the
%% same distribution. (The sources need not necessarily be
%% in the same archive or directory.)
%%
%% The first command in your LaTeX source must be the \documentclass command.
\documentclass[sigconf]{acmart}
\usepackage[true]{anonymous-acm}


%%%% As of March 2017, [siggraph] is no longer used. Please use sigconf (above) for SIGGRAPH conferences.

%%%% As of May 2020, [sigchi] and [sigchi-a] are no longer used. Please use sigconf (above) for SIGCHI conferences.

%%%% Proceedings format for SIGPLAN conferences 
% \documentclass[sigplan, anonymous, review]{acmart}

%%%% Proceedings format for conferences using one-column small layout
% \documentclass[acmsmall,review]{acmart}

%%
%% \BibTeX command to typeset BibTeX logo in the docs
\AtBeginDocument{%
  \providecommand\BibTeX{{%
    \normalfont B\kern-0.5em{\scshape i\kern-0.25em b}\kern-0.8em\TeX}}}

%% Rights management information.  This information is sent to you
%% when you complete the rights form.  These commands have SAMPLE
%% values in them; it is your responsibility as an author to replace
%% the commands and values with those provided to you when you
%% complete the rights form.
\setcopyright{acmcopyright}
\copyrightyear{2021}
\acmYear{2021}
\acmDOI{10.1145/1122445.1122456}


%% These commands are for a PROCEEDINGS abstract or paper.
\acmConference[CEP '21]{Durham'21: ACM Computing Education Practice Conference}{January 07, 2021}{Durham, United Kingdom}
\acmBooktitle{CEP '21: ACM Computing Education Practice Conference,
  {January 07, 2021}{Durham, United Kingdom}
}
\acmPrice{15.00}
\acmISBN{978-1-4503-XXXX-X/18/06}


%%
%% Submission ID.
%% Use this when submitting an article to a sponsored event. You'll
%% receive a unique submission ID from the organizers
%% of the event, and this ID should be used as the parameter to this command.
%%\acmSubmissionID{123-A56-BU3}

%%
%% The majority of ACM publications use numbered citations and
%% references.  The command \citestyle{authoryear} switches to the
%% "author year" style.
%%
%% If you are preparing content for an event
%% sponsored by ACM SIGGRAPH, you must use the "author year" style of
%% citations and references.
%% Uncommenting
%% the next command will enable that style.
%%\citestyle{acmauthoryear}

%%
%% end of the preamble, start of the body of the document source.
\begin{document}

%%
%% The "title" command has an optional parameter,
%% allowing the author to define a "short title" to be used in page headers.
\title{Integrating New Lecturers into the UK Computer Science Community}

%%
%% The "author" command and its associated commands are used to define
%% the authors and their affiliations.
%% Of note is the shared affiliation of the first two authors, and the
%% "authornote" and "authornotemark" commands
%% used to denote shared contribution to the research.
\authoranon{
 \author{Alan Hayes}
 \authornote{All Authors contributed equally}
 \author{James H. Davenport}
 \authornotemark[1]
  \affiliation{%
   \institution{ University of Bath}
   \city{Bath}
   \country{UK}
 }
 \email{a.hayes@bath.ac.uk}
 \email{j.h.davenport@bath.ac.uk}


 \author{Tom Crick}
 \authornotemark[1]
 \affiliation{
   \institution{Swansea University}
   \city{Swansea}
   \country{UK}
 }
 \email{thomas.crick@swansea.ac.uk}


 \author{Alastair Irons}
 \authornotemark[1]
 \affiliation{
   \institution{ Sunderland University}
   \city{Sunderland}
   \country{UK} }
 \email{alastair.irons@sunderland.ac.uk}
 

 \author{Tom Prickett}
 \authornotemark[1]
 \affiliation{
   \institution{ Northumbria University}
   \city{Newcastle upon Tyne}
   \country{UK}
 }
 \email{tom.prickett@northumbria.ac.uk}

}

%%
%% By default, the full list of authors will be used in the page
%% headers. Often, this list is too long, and will overlap
%% other information printed in the page headers. This command allows
%% the author to define a more concise list
%% of authors' names for this purpose.
\renewcommand{\shortauthors}{Trovato and Tobin, et al.}

%%
%% The abstract is a short summary of the work to be presented in the
%% article.
\begin{abstract}
  To Do
  
  The complex and contested demands of learning and teaching in UK\&I
higher education make the early career of an academic challenging ~\cite{Thomas2015} and potentially lonely ~\cite{Foote2009},
especially when balanced against their research aspirations, and wider professional service
commitments. Learning and teaching
development in the UK commonly involves working
towards Fellowship of the Higher Education Academy~\cite{fellowship}
(now known as AdvanceHE), either by direct application or by an
accredited university postgraduate course. Typically, this is
supported by mentoring from within a department. The quality of
learning provided will be promoted in part by the
strength of the community of practice operating within the
department~\cite{Bolander2008} and the communities of practice that
exist at a national and international level~\cite{Thomas2015}.
\end{abstract}

%%
%% The code below is generated by the tool at http://dl.acm.org/ccs.cfm.
%% Please copy and paste the code instead of the example below.
%%
\begin{CCSXML}
<ccs2012>
<concept>
<concept_id>10003456.10003457.10003527</concept_id>
<concept_desc>Social and professional topics~Computing education</concept_desc>
<concept_significance>500</concept_significance>
</concept>
</ccs2012>
\end{CCSXML}

\ccsdesc[500]{Social and professional topics~Computing education}

%%
%% Keywords. The author(s) should pick words that accurately describe
%% the work being presented. Separate the keywords with commas.
\keywords{computer science education, community, induction}


%%
%% This command processes the author and affiliation and title
%% information and builds the first part of the formatted document.
\maketitle


\section{What is it?}	
%%A short description of the practice you're presenting
A short description of the practice you're presenting
\section{Why are you doing it?}
%%What happened before? What is it changing / replacing / improving? What gap is it filling?
What happened before? What is it changing / replacing / improving? What gap is it filling?
\section{Where does it fit?}
%%A short description of your teaching context. You may, for instance, include a description of intake, class size, curriculum sequence; anything that's necessary for others to understand your situation. How do things work at your institution?
A short description of your teaching context. You may, for instance, include a description of intake, class size, curriculum sequence; anything that's necessary for others to understand your situation. How do things work at your institution?
\section{Does it work?}	
%%How do you know? Give some evidence of effectiveness in context.
How do you know? Give some evidence of effectiveness in context.
\section{Who else has done this?}
%%Where did you get the idea from? (If from published reports, please include references). How did you find out about it? Was it easy/hard to adopt? What did you change?
Where did you get the idea from? (If from published reports, please include references). How did you find out about it? Was it easy/hard to adopt? What did you change?
\section{What will you do next?}
%%Will you vary this, or develop it further?
Will you vary this, or develop it further?
\section{Why are you telling us this?}	
%%What is interesting, or useful, about this to someone else?
What is interesting, or useful, about this to someone else?



%% The acknowledgments section is defined using the "acks" environment
%% (and NOT an unnumbered section). This ensures the proper
%% identification of the section in the article metadata, and the
%% consistent spelling of the heading.
\begin{acks}
 
  All the authors' institutions are members of the Institute of Coding (IoC), an initiative funded by the Office for Students (England) and the Higher Education Funding Council for Wales. The BCS Academy of Computing and the IoC has supported the development of this paper
 

\end{acks}

%%
%% The next two lines define the bibliography style to be used, and
%% the bibliography file.
\bibliographystyle{ACM-Reference-Format}
\bibliography{NewLecturers}

%%
%% If your work has an appendix, this is the place to put it.
%%\appendix

\balance
\end{document}
\endinput
%%
%% End of file `sample-sigconf.tex'.
