%%%% anonymous-acm.sty  
%% version 1.0 09-May-2020   
%% Maintained by Brett A. Becker: brett.becker@ucd.ie

% This work may be distributed and/or modified under the
% conditions of the LaTeX Project Public License, either 
% version 1.3 of this license or any later version.
% The latest version of this license is in
% http://www.latex-project.org/lppl.txt
% and version 1.3 or later is part of all distributions of LaTeX
% version 2005/12/01 or later.

% This program is distributed in the hope that it will be useful 
% but WITHOUT ANY WARRANTY; without even the implied warranty of 
% MERCHANTABILITY or FITNESS FOR A PARTICULAR PURPOSE.

% designed to be used with acmart.cls - required for full functionality

%% Identification
%% The package identifies itself and the LaTeX version needed
\ProvidesPackage{anonymous-acm}
\NeedsTeXFormat{LaTeX2e}


% Define anonymous condition
\newif\ifAnonCondition\AnonConditiontrue

% Declare options
\DeclareOption{true}{\AnonConditiontrue}
\DeclareOption{false}{\AnonConditionfalse}
\DeclareOption*{\PackageWarning{anonymous-acm}{Unknown ‘\CurrentOption’}}
\ProcessOptions\relax

% Anonymous Authors
\newcommand{\authoranon}[1]{#1}
\ifAnonCondition
\renewcommand{\authoranon}[1]{
\author{Anonymous Author(s)}
}
\else
\renewcommand{\authoranon}[1]{#1}
\fi

% Anonymous Arbitrary Text
\newcommand{\textanon}[2]{#1}
\ifAnonCondition
\renewcommand{\textanon}[2]{\ifstrequal{}{#2}{<text removed for peer review>}{#2}}
\else
\renewcommand{\textanon}[2]{#1}
\fi

% Anonymous Links
\newcommand{\linkanon}[2]{\href{#1}{#2}}
\ifAnonCondition
\renewcommand{\linkanon}[2]{<anonymous link>}
\else
\renewcommand{\linkanon}[2]{\href{#1}{#2}}
\fi

\newcommand{\textlinkanon}[2]{\href{#1}{#2}}
\ifAnonCondition
\renewcommand{\textlinkanon}[2]{#2}
\else
\renewcommand{\textlinkanon}[2]{\href{#1}{#2}}
\fi

% Anonymous Citations & References
\newcommand{\citeanon}[2][\ ]{\cite[#1]{#2}}
\ifAnonCondition
\renewcommand{\citeanon}[2][\ ]{[anonymous]}
\else
\renewcommand{\citeanon}[2][@]{\ifstrequal{@}{#1}{\cite{#2}}{\cite[#1]{#2}}}
\fi

% Anonymous Acknowledgments
\ifAnonCondition
\excludecomment{acks}
\fi

%%
%% This is file `sample-sigconf.tex',
%% generated with the docstrip utility.
%%
%% The original source files were:

%%
%% samples.dtx  (with options: `sigconf')
%% 
%% IMPORTANT NOTICE:
%% 
%% For the copyright see the source file.
%% 
%% Any modified versions of this file must be renamed
%% with new filenames distinct from sample-sigconf.tex.
%% 
%% For distribution of the original source see the terms
%% for copying and modification in the file samples.dtx.
%% 
%% This generated file may be distributed as long as the
%% original source files, as listed above, are part of the
%% same distribution. (The sources need not necessarily be
%% in the same archive or directory.)
%%
%% The first command in your LaTeX source must be the \documentclass command.
\documentclass[sigconf]{acmart}
\usepackage[false]{anonymous-acm}
\usepackage{paralist}
\settopmatter{authorsperrow=3}

%%%% As of March 2017, [siggraph] is no longer used. Please use sigconf (above) for SIGGRAPH conferences.

%%%% As of May 2020, [sigchi] and [sigchi-a] are no longer used. Please use sigconf (above) for SIGCHI conferences.

%%%% Proceedings format for SIGPLAN conferences 
% \documentclass[sigplan, anonymous, review]{acmart}

%%%% Proceedings format for conferences using one-column small layout
% \documentclass[acmsmall,review]{acmart}

%%
%% \BibTeX command to typeset BibTeX logo in the docs
\AtBeginDocument{%
  \providecommand\BibTeX{{%
    \normalfont B\kern-0.5em{\scshape i\kern-0.25em b}\kern-0.8em\TeX}}}

\copyrightyear{2021} 
\acmYear{2021} 
\setcopyright{acmcopyright}\acmConference[CEP '21]{Computing Education Practice 2021}{January 7, 2021}{Durham, United Kingdom}
\acmBooktitle{Computing Education Practice 2021 (CEP '21), January 7, 2021, Durham, United Kingdom}
\acmPrice{15.00}
\acmDOI{10.1145/3437914.3437977}
\acmISBN{978-1-4503-8959-4/21/01}


%%
%% Submission ID.
%% Use this when submitting an article to a sponsored event. You'll
%% receive a unique submission ID from the organizers
%% of the event, and this ID should be used as the parameter to this command.
%%\acmSubmissionID{123-A56-BU3}

%%
%% The majority of ACM publications use numbered citations and
%% references.  The command \citestyle{authoryear} switches to the
%% "author year" style.
%%
%% If you are preparing content for an event
%% sponsored by ACM SIGGRAPH, you must use the "author year" style of
%% citations and references.
%% Uncommenting
%% the next command will enable that style.
%%\citestyle{acmauthoryear}

%%
%% end of the preamble, start of the body of the document source.
\begin{document}

%%
%% The "title" command has an optional parameter,
%% allowing the author to define a "short title" to be used in page headers.
\title{Supporting Early-Career Academics in the\\UK Computer Science Community}

%%
%% The "author" command and its associated commands are used to define
%% the authors and their affiliations.
%% Of note is the shared affiliation of the first two authors, and the
%% "authornote" and "authornotemark" commands
%% used to denote shared contribution to the research.
\authoranon{
 
 
 \author{Tom Crick}
 \authornote{{\emph{N.B.}} all authors contributed equally to this paper}
 \affiliation{
   \institution{Swansea University}
   \city{Swansea}
   \country{UK}
 }
 \email{thomas.crick@swansea.ac.uk}

  \author{James H. Davenport}
 \authornotemark[1]
  \affiliation{%
   \institution{ University of Bath}
   \city{Bath}
   \country{UK}
 }
 \email{j.h.davenport@bath.ac.uk}

  \author{Alan Hayes}
\authornotemark[1]
\affiliation{%
	\institution{ University of Bath}
	\city{Bath}
	\country{UK}
}
\email{ah347@bath.ac.uk}

 \author{Alastair Irons}
 \authornotemark[1]
 \affiliation{
   \institution{ Sunderland University}
   \city{Sunderland}
   \country{UK} }
 \email{alastair.irons@sunderland.ac.uk}
 

 \author{Tom Prickett}
 \authornotemark[1]
 \affiliation{
   \institution{ Northumbria University}
   \city{Newcastle upon Tyne}
   \country{UK}
 }
 \email{tom.prickett@northumbria.ac.uk}
 
 
}


%%
%% By default, the full list of authors will be used in the page
%% headers. Often, this list is too long, and will overlap
%% other information printed in the page headers. This command allows
%% the author to define a more concise list
%% of authors' names for this purpose.
\renewcommand{\shortauthors}{Crick et al.}

%%
%% The abstract is a short summary of the work to be presented in the
%% article.
\begin{abstract}
  The early career of a computer science academic in the United
Kingdom (UK) -- as with most other disciplines -- is challenging in
terms of balancing research aspirations, learning and teaching
responsibilities, wider academic service commitments, as well as their
own professional development. In terms of learning and teaching
development, this commonly involves working towards Fellowship of the
Higher Education Academy (now known as Advance HE), either by direct
application or via successful completion of an accredited
institutional taught postgraduate course. Typically, if a course is
required (often as part of their academic probation), the focus will
be general higher education learning and teaching pedagogy rather than
specifically focused on computer science and cognate areas. The formal
institutional course requirements are normally supplemented by
mentoring from within their department from experienced academic
colleagues. Thus, the quality of development for an early-career
academic will be enhanced in part by the strength of the community of
practice operating within the department and the communities of
practice that exist at a national and international level, often
through professional bodies, learned societies and sub-disciplinary
groupings. This paper presents the work-in-progress to address some of
these structural, cultural and community challenges at both the
institutional and national level in the UK, based on empirical themes
collected from a workshop held at UKICER'20. We identify a number of
specific actions and recommendations to supplement the current formal
institutional requirements with enhanced national-level academic
practice support and professional development, alongside local and
regional professional mentoring.
\end{abstract}

%%
%% The code below is generated by the tool at http://dl.acm.org/ccs.cfm.
%% Please copy and paste the code instead of the example below.
%%
\begin{CCSXML}
<ccs2012>
   <concept>
       <concept_id>10003456.10003457</concept_id>
       <concept_desc>Social and professional topics~Professional topics</concept_desc>
       <concept_significance>500</concept_significance>
       </concept>
 </ccs2012>
\end{CCSXML}

\ccsdesc[500]{Social and professional topics~Professional topics}

% \begin{CCSXML}
% <ccs2012>
% <concept>
% <concept_id>10003456.10003457.10003527</concept_id>
% <concept_desc>Social and professional topics~Computing education</concept_desc>
% <concept_significance>500</concept_significance>
% </concept>
% </ccs2012>
% \end{CCSXML}

%\ccsdesc[500]{Social and professional topics~Computing education}

%%
%% Keywords. The author(s) should pick words that accurately describe
%% the work being presented. Separate the keywords with commas.
\keywords{Early-career academics, community of practice, professional development, computer science education}


%%
%% This command processes the author and affiliation and title
%% information and builds the first part of the formatted document.
\maketitle

%%Comment from AI Not sure in which section- but do we want to play up the consultations we have undertaken so far – particularly the work Alan has done in reaching out to colleagues and including positive and negative feedback ? Maybe bring in the work that James identified with the Physics folk in this discussion ??
\section{What is it?}	
%%A short description of the practice you're presenting
This paper outlines the work-in-progress to address the structural,
cultural and community challenges at both the institutional and
national level in the UK. It reports progress from a project
exploring, developing and delivering an initiative to supplement
institution-based professional development and training available to
early-career computer science academics, with an offering of regional
and national activities. It describes the work undertaken to date to
research, co-design and develop a training course for those new to
teaching computer science in the UK.

This initiative began via sector-led discussions regarding how the computer science education community could be better supported at \textanon{BCS Academy of Computing Board}{BLINDED} and \textanon{the Institute of Coding}{BLINDED}. This was followed by the recruitment of a steering group of computer science academics, from across a representative range of UK universities. The academic career profile and experience of the academics involved is deliberately diverse; it includes junior academics to more senior academics, as well as a number of members of the professoriate and senior members of \textanon{BCS Academy of Computing}{BLINDED}. \begin{comment}
content...
The group also includes a number of computer science academics who have been recognised as National Teaching Fellows by Advance HE. 
\end{comment}
The views of this steering committee were combined with the outcomes of a workshop at the \textanon{ACM UK \& Ireland Computing Education Research Conference 2020 (UKICER'20)}{BLINDED Location}, to formulate a pilot training course. The training course will be first run in December 2020, so initial findings and progress can be shared at this conference. This paper presents an overview of the co-design of the trial run of the training programme, as well as the key outcomes from the workshop which took place in September 2020.

% \citeanon{IOC}
% \citeanon{BCSAcademy}
% \citeanon{ACMUKICER}

\section{Why are you doing it?}
%%What happened before? What is it changing / replacing / improving? What gap is it filling?
There are a number of disciplinary challenges which are well
documented within computer science education, including substantial
changes in national curricula and
qualifications~\cite{brown-et-al-toce2014}. Attrition and failure
rates can be high, learning programming for perhaps the first time
remains a significant
hurdle~\cite{davenport-et-al:latice2016,murphy-et-al:programming2017,simon-et-al:sigcse2018},
with a range of issues impacting failure
rates~\cite{Watson:2014:FRI:2591708.2591749,prickett-et-al:iticse2020}. Student
satisfaction as measured by national surveys is reported as commonly
below that of other disciplines~\cite{Sinclair2015}; concerns related
to the employment prospects of graduates have been reported as
inferior to other disciplines, especially across
STEM~\cite{shadbolt2016shadbolt}. Effectively addressing these
disciplinary challenges involves a number of evolving pedagogic
approaches, as well as wider system and societal issues. To become a
confident and capable computer science academic (from a learning \&
teaching perspective) requires maintaining currency in terms of
specialist subject knowledge (and how their own research informs their
practice), but also fluency in how to respond to the
education challenges the discipline presents.

The complex demands of UK higher education
make the early career of an academic challenging~\cite{Thomas2015}
and potentially lonely~\cite{Foote2009}, especially when balanced
against their research aspirations (or indeed, institutional benchmarks), and wider
professional service commitments. Professional development in the UK
commonly involves working towards Fellowship of the Higher Education
Academy
(FHEA)\footnote{\url{https://www.advance-he.ac.uk/fellowship}}, either
by direct application to Advance HE or by an accredited university
postgraduate course. Typically, this is supported by mentoring from
within a department. The quality of learning provided will be promoted
in part by the strength of the community of practice operating within
the department~\cite{Bolander2008} and the communities of practice
that exist at a national and international level~\cite{Thomas2015}.

Simultaneously mastering the challenges of ``learning the craft''
and those of how to become an effective and successful academic can be
fraught. However, some mechanisms already exist to better support this
professional development. For a number of years, mentoring has been
recognised as a important mechanism in the development of academics,
with formal mentoring now common as part of the process of on-boarding
early-career academics. Commonly this takes the form of an experienced
academic providing support for an early-career colleague. Mentoring
has been highlighted as part of a set of activities help diversify the
staff base~\cite{Golubchik2018}. While such mechanisms can be very
effective, limiting guidance and support to one source rather than a
wider and more diverse community can be
limiting~\cite{deJanasz}. Informal mentoring can also be very
effective, but it is not without issue. For example, it has been
reported that women can feel marginalised from informal mentoring
partnerships~\cite{gibson2006mentoring}. In the UK, Black, Asian and
Minority Ethnic (BAME) colleagues are ``\emph{poorly represented in
both senior academic and university leadership roles: of 19,000 people
employed as professors in the United Kingdom, only 400 are BAME
women. In a typical gathering of 100 professors, 90 would be white and
there would be just two BAME women.}''
\cite[p.~I]{UniversitiesUK19}. There are a plethora of issues that
continue to reinforce aspects of this of this
inequality~\cite{arday20}. Policy recommendations have been made
(including new models for mentoring and the establishing of support
networks~\cite{bhopal2014experiences}), and initiatives developed and
executed (including Athena
Swan\footnote{\url{https://www.advance-he.ac.uk/equality-charters/athena-swan-charter}}
and
the Race Equality Charter\footnote{\url{
    https://www.advance-he.ac.uk/charters/race-equality-charter}}. Equally,
research suggests that care needs to be taken not to conflate issues
of race and gender~\cite{Bhopal19}, or some of the challenges being
faced may not be addressed. Research related to cross-gender or
cross-racial mentoring is relatively rare, however it has been
reported that cross-racial mentoring between faculty and students can
work if an ally development model is
employed~\cite{reddick2016don}. Clearly, ally work can be challenging and
will require continual reflection, perseverance and careful
consideration of power and positions in response to
repression~\cite{Patton2015}. However the distance and absence of
direct power structures between universities may make it more
straightforward to function as an ally in a cross-university mentoring
relationship which is unburdened by managerial or organisational
concerns. Such mentoring relationships are also likely to be
educational for the mentors, who would also be promoting and giving
back to the community. Ideally, the mentoring will evolve into a
longer term relationship which are mutually beneficial and help to
promote and facilitate work-based learning for the academics involved.

% This paper summaries an initiative to date to explore the potential to supplement the existing initiatives with further regional and national-level training and development opportunities for early-career computer science academics.


\section{Where does it fit?}
%%A short description of your teaching context. You may, for instance, include a description of intake, class size, curriculum sequence; anything that's necessary for others to understand your situation. How do things work at your institution?
In the UK, early-career academics will have had varied prior
experience; including: postdocs taking up their first academic
appointment at the start of their academic career; postdocs who are
teaching as part of their career development in preparation for
applying for a permanent academic position; PhD students, aspiring to
an academic career, with significant teaching responsibilities;
established international academics new to the UK HE sector; junior
academics who have been in a temporary post for a few years; those
moving from FE into their first HE job; those moving from industry
into their first HE job; and CS academics at institutions who do not
currently specify that their lecturing staff complete a PG teaching
qualification.

In any given year, the number of individuals across these profiles
will vary; but with over one hundred higher education institutions in
the UK that teach computing at undergraduate or postgraduate level,
there is usually at least one faculty member in this position, hence
there exists a potentially large group of individuals who could
benefit from this proposed professional development programme.

% For the pilot phase of this programme (which will run in December 2020), a training course will be run for early-career academics based at the institutions of those who are members of the steering committee. The training programme follows a blended approach consisting of synchronous online sessions supported with the availability of the material on a supporting website. The programming includes live virtual sessions presented by colleagues from UK computer science departments. The sessions will be practical in nature, and includes pedagogical groundings, the sharing of best practice, how to developing good practice, and (perhaps more importantly) the sharing of ``what to avoid doing''. The intention is to supplement the programme with the provision of cross-institution mentoring at the local and regional level.

For the pilot phase of this programme (which will run in December
2020), a training course will be run for invited early-career
academics based at the institutions of the members of the steering
committee. The programme includes live online sessions presented by
colleagues from UK computer science departments. For interested
attendees, the intention is to supplement the programme with the
provision of cross-institution mentoring.

\section{Does it work?}	
%%How do you know? Give some evidence of effectiveness in context.

%% Comment from AI - Worth emphasising  that the sub sections in 4 are as a result of the workshop ? Have we ranked them based on number of responses and / or discussion ???Maybe worth saying that there were xxxx post it responses and we identified key themes ?? Worth saying that Tom P facilitated the data gathering and encouraged discussion of common points ?? (action research ?? )
%% comment from TP - I have tried to update to include this. I didnt do this with the degree of rigour suggested above. What I did was analyse for the key themes. e.g. slightly informal open coding analysis. I didnt do any counting. I didnt think I was supposed to with open coding analysis e.g. what are the themes
%  \citeanon{ACMUKICER}
%JHD next line said 2021, but I think we meant 2020.
The workshop  at \textanon{UKICER'20}{BLINDED Location}, focused upon addressing three research questions:
\begin{compactenum}[i)]
\item What current and future opportunities are there to engage new CS faculty with a UK Computer Science Education Community of Practice?
\item What is the potential to supplement institutional academic/research development opportunities with national development opportunities?
\item How can we continue to raise the profile and value of new research faculty members engaging with computer science education research?
\end{compactenum}
Engagement in computer science education {\emph{research}} is not the same as engaging in computer science education {\emph{practice}}; however it can be a reasonable proxy for understanding the wider domain challenges. At the UKICER'20 workshop, the attendees engaged in a facilitated session to encourage discussion of key issues, providing responses to a set of questions using an online collaborative environment. The input from the 21 attendees generated a number of insights into the challenges faced by early-career computer science academics. After the workshop, the online interactions and responses were analysed using open coding analysis in order to identify key themes:

\noindent\textbf{Inconsistent institutional support.}
Variable institutional-level support was identified. Some colleagues reported little support for new faculty with respect to learning and teaching; others reported postgraduate training programmes or even higher apprenticeships.

\noindent\textbf{Competing priorities and resource constraints.}
Balancing alternative pressures (especially research) came out strongly as a key challenge in increasing interest in computer science education. Hence any training course has to be designed with direct benefit for attendees. 

\noindent\textbf{Changing negative perceptions.}
A few participants commented less favourable regarding how computer science education was seen in their home universities, together with a wish to support initiatives the raised its profile and perceived value.

\noindent\textbf{Islands of excellence.}
There are a number of intuitions within the UK with significant (or developing) research interests in computer science education (for example: \textanon{Glasgow, KCL, Kent, Napier}{X,Y,Z}); the majority of others institutions lacked such support or structures. It was suggested that in some cases further collaboration with Education departments may also be of benefit. Elsewhere, some colleagues reported the absence of computer science education research groups and the related infrastructure to promote the importance of computer science education.

\noindent\textbf{Local and other initiatives.}
A number of local initiatives exist, together with a growing number of national initiatives (as discussed in the following section); this suggests there is an appetite for disciplinary-specific professional development initiatives such as this one.

\noindent\textbf{A developing UK community.}
The national computer science education community was seen as a very positive development and further initiatives that intended to strengthen or grow the community appeared to be well received.

\section{Who else has done this?}
%%Where did you get the idea from? (If from published reports, please include references). How did you find out about it? Was it easy/hard to adopt? What did you change?
It is reported that the most commonly employed mechanism for
continuing professional development (CPD) is conversations between
peers~\cite{King2004}. A number of communities exist to support these
conversations at the local, national and international
level. Internationally this development is supported by the groupings
such as ACM Special Interest Group in Computer Science Education
(SIGCSE) or the IEEE Education Society. These international
professional bodies run a number of conferences to promote discussion
and dialog as well as dissemination of ideas. The UK \& Ireland SIGCSE
Chapter further promotes the community-building by running annual
Computing Education Practice (CEP) and Computing Education Research
(UKICER) conferences. However, whilst
early-career academics are welcome to attend, the conferences and
seminars are not directly aimed a the needs of early-career academics
who may in any case find it challenging to obtain funding for
attendance. The ACM UK SIGCSE Chapter also holds Twitter meets and
runs a computer science education research journal club. Whilst
early-career computer science academics are very much welcome to join
in these discussions, they are not the primary target market. Some
universities have centres which further promote these conversations,
such as the Centre for Computing Science Education at the University
of
Glasgow\footnote{\url{https://www.gla.ac.uk/research/az/ccse/}}. 
And as identified in both the 2012 and 2017 Royal Society reviews of
computing education in the UK, there is a small (but growing) base in
UK universities. In summary, the international and national
Initiatives help promote a community that experienced academics can
access to provide CPD; however the CPD for early-career academics is
less developed, particularly for those at institutions who do not have
established centres or CSE research groups.

% \cite{SIGCSE}
% \cite{IEEEES}
% \cite{UKI-SIGCSE}
% \cite{CEP}
% \cite{UKICER}
% Conversations are also promoted by the Raspberry PI Foundation via it Symposium \cite{PI_SYM} and online seminars \cite{PI_Sem}.
% for example the Edinburgh Napier Centre for Computer Education Research \cite{Napier}

In addition to training programmes run by individual Universities, in
the UK, Advance HE runs training programmes for academics at different
stages of their career, alongside their legacy ``subject
centres''. However this training is not discipline-specific. The UK's
Council of Professors and Heads of Computing (CPHC) run occasional
workshops in a variety of issues. The most pertinent offering for new
faculty has been the ``Chair in 10 Years'' workshop which is aimed at
facilitating medium-to-long-term academic career planning. Whilst this
is a well received contribution, it does not directly support the
development needs of new faculty with respect to high-quality
learning, teaching and assessment.

As well as generic pedagogic challenges, most disciplines including
computer science have a range of discipline specific challenges. The
Institute of Mathematics and its Applications (IMA) for a number of
years has ran a training course intended to welcome new faculty to the
community and help address key discipline-related pedagogic
challenges~\cite{IMA}.

% \cite{HEATraining}

\section{What will you do next?}
%%Will you vary this, or develop it further?

The recent mapping and analysis highlights that there is space within the UK computer science community of practice to establish a new training programme for early-career computer science academics that augments what is already provided by individual institutions. However discussions within the steering group of academics highlight a few questions that would benefit from being answered in order to appropriately design the new programme including:
\begin{compactitem}
	\item \textbf{Motivation to attend.} Early career colleagues are extremely busy so the course needs to be genuniely helpful and the benefits from attending need to be very clear
	\item \textbf{Building a community.} The course needs to facilitate the growing of the community between the attendees and links to the wider community of practice in the UK.
	\item \textbf{Professional alignment.} Early day discussions have taken place with both Advanced HE and the BCS, The Chartered Institute for IT. Both organisations are supportive of the initiative in principle. Discussions are ongoing regarding what the best professional allignment will be.
	\item \textbf{Delivery mode.} At the time of writing the only possible manner is online. However other modes might be preferable in the longer term.
	\item \textbf{Prototype course.} All these questions require the input of early career colleagues to resolve. These questions will be explore at a training course in December 2020. This course will provide: guidance upon issues the steer group believes are critical; a panel which will seek to address three questions from each attendees; and a workshop to explore the development needs of early career computer science academics. 
\end{compactitem}
 

\begin{comment}


The discussions to date have led to the co-design of a training programme shown in Table~\ref{tab:table1}. The programme will be delivered over five half-day blocks:
\begin{table}[h]
%%\begin{tabular}{|l|p{200}|}
\begin{tabular}{|p{0.3in}|p{2.8in}|}
\hline
Day   & Subject                                                      \\ \hline
1 & Effective CS pedagogy and assessment      \\ \hline
2 & Management of learning and teaching for CS   \\ \hline
3 & UK higher education landscape, policy and governance                 \\ \hline
4 & UK (and international) research environment and support structures \\ \hline
5 & Emerging trends in CS learning, teaching and assessment         \\ \hline
\end{tabular}
\caption{Course Content Overview}
\label{tab:table1}
\end{table}
%%Comment from AI - maybe something about in addition we are looking at mentoring scheme ??
%%And maybe something about ongoing CPD – these days could be opened up to established colleagues (or is that a whole new paper). But also ongoing CPD for those new colleagues who complete the 5
%%Something back to community of practice – peer led development and WBL for academics ??
This programme will be delivered for the first time in December 2020. The initial delivery of the programme will be via a blended learning model, with attendees completing activities as well as attending a series of interactive and collaborative webinars. As part of the pilot programme, the intention will be to establish models of cross-university mentoring at the local and regional level. The group of academics involved in the development of this initiative have broad learning, teaching and assessment expertise within the field of computer science, in both the UK and internationally. This breadth of expertise will enable mentoring across diverse learning and teaching responsibilities. Following evaluation of this pilot phase, the programme will be made widely available into 2021 and beyond. Opting for a blended learning approach takes into consideration the longer-term impacts of the COVID-19 pandemic on the UK higher education sector, especially considering its specific impact on the computer science education community~\cite{crick-et-al:ukicer2020}; however, further consideration will be given to future face-to-face delivery models.
\end{comment}

\section{Why are you telling us this?}	
%%What is interesting, or useful, about this to someone else?
The UK's computer science education community of practice has evolved and developed rapidly in recent years, which presents the opportunity to review and celebrate the current achievements. As part of this review, this project has identified a significant opportunity to establish: {\emph{(i)}} a national training programme for those new to teaching computer science in the UK, which will augment institutional professional development and training opportunities; and {\emph{(ii)}} a national mentoring programme for those new to teaching computer science in the UK, supplemented by local and regional networks. These parallel initiatives are intended to adopt the ally model which is facilitated by the cross-institutional nature of the initiative. In doing so, as well as supporting the continued focus on high-quality learning, teaching and assessment in computer science (with a specific focus on effective pedagogic approaches), and the further development of the UK's computer science education community of practice, it is intended to provide attendees an ally who can help them address sector and system-level racial and gender inequalities that persist in UK higher education (as well as the wider IT profession\footnote{Supporting work by BCS, The Chartered Institute for IT on diversity and inclusion: \url{https://www.bcs.org/policy-and-influence/diversity-and-inclusion/}}.

The initiative presents a range of opportunities for those at various points of their academic career. For those at an early career stage, the initiative presents a further source of valuable professional development and extension of their professional networks outside of their institution. For those with more experience, as the initiative develops so will opportunities to become involved and support activities both within their institution, as well as across the sector. If the approach is effective, and the community wishes to see it extended then the involvement of experienced colleagues in its development, delivery and enhancement as mentors will be critical to ensure wide access and engagement.

%% The acknowledgments section is defined using the "acks" environment
%% (and NOT an unnumbered section). This ensures the proper
%% identification of the section in the article metadata, and the
%% consistent spelling of the heading.
% \begin{acks}
%  All the authors' institutions are members of the Institute of Coding (IoC), an initiative funded by the Office for Students (England) and the Higher Education Funding Council for Wales. The BCS Academy of Computing has also supported the development of this work.
% \end{acks}

%%
%% The next two lines define the bibliography style to be used, and
%% the bibliography file.
\bibliographystyle{ACM-Reference-Format}
\bibliography{NewLecturers}

%%
%% If your work has an appendix, this is the place to put it.
%%\appendix

\balance
\end{document}
\endinput
%%
%% End of file `sample-sigconf.tex'.
