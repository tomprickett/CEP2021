\font\manual=manfnt
\def\dbend{{\manual\char127}}
\def\eqq{{\buildrel?\over=}}
\def\C{\mathbf{ C}}
\def\N{\mathbf{ N}}
\def\Q{\mathbf{ Q}}
\def\R{\mathbf{ R}}
\def\Z{\mathbf{Z}}
\def\F{\mathbf{F}}
\def\softO{\tilde{\cal O}}
\def\O{{\cal O}}
%\def\N{{\bf N}}
%\def\Q{{\bf Q}}
%\def\R{{\bf R}}
%\def\Z{{\bf Z}}
\def\action#1{\hfil\rlap{\bf#1}\break}
\def\pcite#1{[#1]}
\def\RootOf{\mathop{\rm RootOf}\nolimits}
\def\r#1#2{``#1''$\rightarrow$``#2''}
\documentclass{article}
\bibliographystyle{alphaurl}
\usepackage[hyphens]{url}
%\usepackage{enumitem}
\usepackage{pdfpages}
\usepackage{verbatim}
\usepackage{hyperref}
\usepackage{combelow}
\usepackage[show]{ed}
\usepackage{graphicx}
\newtheorem{proposition}{Proposition}
\newtheorem{theorem}{Theorem}
\newtheorem{lemma}{Lemma}
\newtheorem{corollary}{Corollary}
\newtheorem{definition}{Definition}
\newtheorem{notation}{Notation}
\newtheorem{example}{Example}
\newtheorem{problem}{Problem}
\def\decision#1{\par{\bf #1}}
\def\action#1{\hfil\rlap{\bf#1}\break}
%\url{https://assessment.bath.ac.uk/admin#grading/overview/78727640}
\pagestyle{empty}
\begin{document}
\author{Notes by J.H.Davenport}
\title{REshaping Assessment Excellence: Pandemic and the New Now}
\date{31 July 2022}
\maketitle
It is hard to separate teaching from assessment, as assessment drives much student motivation.

\section{An Example}
The University of Bath has operated a semester system for practically all programmes, with examination assessments in, essentially, January and May.
\begin{description}
\item[Before]Covid-19, the only option for an examination was Trad-C or Trad-O, nearly all of two hours duration. Though this was not always possible, the aim was that a student should only have one examination per day.
\item[May 2020]Given the timing, {\bf Interim} was the only practicable solution. Since many solutions were ``at home'' across the world, the examinations were still aimed at taking 2 hours, but all students were given a 24-hour window in which to do them. Very few staff had experience in ``open book'' examinations, and certainly not when the whole Internet was an open book. Now we needed to have one examination per day, which stretched the examination period.
\item[January 2021]We moved to Electronic-O.
\item[May 2021]
\item[January 2022]
\item[May 2022]
\end{description}
\def\r{$\rightarrow$}
\section{Models}
\begin{description}
\item[Trad-C]A university-managed invigilated examination in an ``examination hall'', with no technology or other resources except the question paper.
\item[Trad-O]As above but students can bring in specified paper resources: often called ``open book''. That specification could be very precise: ``A clean copy of the fifth edition of \dots'', or as vague as ``a binder of own notes''.  In practice invigilators found this hard to enforce in a large examination.
\item[Interim]Use a Virtual Learning Environment to deliver an examination paper, and collect answers. There are no technological constraints on the help students could acquire.
\item[Electronic-C]
\item[Electronic-O]
\end{description}
\bibliography{../../../jhd}
\end{document}
\section{}
\begin{description}
\item[Q]
\item[A]
\end{description}
\begin{enumerate}
\item 
\end{enumerate}
\begin{quote}
\end{quote}
\begin{itemize}
\item
\end{itemize}
\begin{example}
\end{example}
\begin{definition}
\end{definition}
\begin{theorem}
\end{theorem}
\begin{description}
\item[Theme 1]
\item[Theme 2]
\item[Theme 3]
\item[Theme 5]
\item[Theme 6]
\item[Monitoring]
\end{description}
\begin{enumerate}
\item 
\end{enumerate}
\begin{quote}
\end{quote}
