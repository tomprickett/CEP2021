\font\manual=manfnt
\def\dbend{{\manual\char127}}
\def\eqq{{\buildrel?\over=}}
\def\C{\mathbf{ C}}
\def\N{\mathbf{ N}}
\def\Q{\mathbf{ Q}}
\def\R{\mathbf{ R}}
\def\Z{\mathbf{Z}}
\def\F{\mathbf{F}}
\def\softO{\tilde{\cal O}}
\def\O{{\cal O}}
%\def\N{{\bf N}}
%\def\Q{{\bf Q}}
%\def\R{{\bf R}}
%\def\Z{{\bf Z}}
\def\action#1{\hfil\rlap{\bf#1}\break}
\def\pcite#1{[#1]}
\def\RootOf{\mathop{\rm RootOf}\nolimits}
\def\r#1#2{``#1''$\rightarrow$``#2''}
\documentclass{article}
\bibliographystyle{alphaurl}
\usepackage[hyphens]{url}
%\usepackage{enumitem}
\usepackage{pdfpages}
\usepackage{verbatim}
\usepackage{hyperref}
\usepackage{combelow}
\usepackage[show]{ed}
\usepackage{graphicx}
\newtheorem{proposition}{Proposition}
\newtheorem{theorem}{Theorem}
\newtheorem{lemma}{Lemma}
\newtheorem{corollary}{Corollary}
\newtheorem{definition}{Definition}
\newtheorem{notation}{Notation}
\newtheorem{example}{Example}
\newtheorem{problem}{Problem}
\def\decision#1{\par{\bf #1}}
\def\action#1{\hfil\rlap{\bf#1}\break}
%\url{https://assessment.bath.ac.uk/admin#grading/overview/78727640}
\pagestyle{empty}
\begin{document}
\author{Notes by J.H.Davenport}
\title{REshaping Assessment Excellence: Pandemic and the New Now}
\date{31 July 2022}
\maketitle
It is hard to separate teaching from assessment, as assessment drives much student motivation. Though there has been much debate abuot the various froms of assesment, and the role of technology in assessment, actaul change has been slow pre-Covid, as evidenced in this statement \cite{WilliamsWong2009a}.
\begin{quote}
Indeed, the final examination is a university institution that would appear to be off-limits as far as innovation is concerned. To put this into context, while faculty andstudents alike will not stray too far from a computer as they go about their dailybusiness, it is still the norm for examinations to be conducted using pen and paper. Doesthis imply, therefore, that some element of modern learning theory might be sacrificedif it were abandoned in favour of some alternative instrument? Or, given it is still themost commonly administered summative assessment instrument in universities today,is there some other special intrinsic value attached to a closed-book, invigilated examthat justifies its continued use
\end{quote}

\section{An Example}
The University of Bath has operated a semester system for practically all programmes, with examination assessments in, essentially, January and May.
\begin{description}
\item[Before]Covid-19, the only option for an examination was Trad-C or Trad-O, nearly all of two hours duration. Though this was not always possible, the aim was that a student should only have one examination per day. It was possible to disguise a {\bf TakeHome-O-U} examination as ``coursework''.
\item[May 2020]Given the timing, {\bf Interim} was the only practicable solution. Since many solutions were ``at home'' across the world, the examinations were still aimed at taking 2 hours, but all students were given a 24-hour window in which to do them. Very few staff had experience in ``open book'' examinations, and certainly not when the whole Internet was an open book. Now we needed to have one examination per day, which stretched the examination period.
\item[January 2021]We moved to Electronic-O.
\item[May 2021]
\item[January 2022]
\item[May 2022]
\end{description}
\def\r{$\rightarrow$}
\section{Models}
We use the British English word ``invigilated'' --- the corresponding American word is ``proctored''.
\begin{description}
\item[Trad-C]A university-managed invigilated examination in an ``examination hall'', with no technology or other resources except the question paper.
\item[Trad-O-R]As above but students can bring in specified paper resources: often called ``open book''. That specification could be very precise: ``A clean copy of the fifth edition of \dots'', or as vague as ``a binder of own notes''.  In practice invigilators found this hard to enforce in a large examination.
\item[Trad-O-U]As above with no restrictions on what an be brought it (on paper).
\item[TakeHome-O-U]In a take-home examination, the student is given the question paper, and has to bring the answer back later (generally 24-hours). \cite{bengtsson2019a} is a useful survey of these.
\item[Interim]Use a Virtual Learning Environment to deliver an examination paper, and collect answers. There are no technological constraints on the help students could acquire.
\item[Electronic-C]A university-managed examination, generally using a specific software platform\footnote{Bath used Inspera, but the precise choice is probably irrelevant.} 
\item[Electronic-O]
\end{description}
\section{Cheating etc.}
\cite{Dickinson2022a} reports a small ($N=900$) survey \cite{AcademicAppeals2022a}\footnote{One item from this not reproduced is ``Of those students who admitted to cheating, only a very small minority --- 5\% --- were caught by their institutions''. } of UK students.
\begin{quote}
  The numbers suggest that 1 in 6 students in the UK have cheated in online exams this academic year. Over half of those surveyed knew people who had cheated in online assessments. Almost 8 out 10 believed that it was easier to cheat in online exams than in exam halls, and the methods for cheating were often laughably rudimentary – including calling or messaging friends for help during the exam, using google to search for answers on a separate device, or asking parents to read through answers prior to submission.
\end{quote}
\bibliography{../../../jhd}
\end{document}
\section{}
\begin{description}
\item[Q]
\item[A]
\end{description}
\begin{enumerate}
\item 
\end{enumerate}
\begin{quote}
\end{quote}
\begin{itemize}
\item
\end{itemize}
\begin{example}
\end{example}
\begin{definition}
\end{definition}
\begin{theorem}
\end{theorem}
\begin{description}
\item[Theme 1]
\item[Theme 2]
\item[Theme 3]
\item[Theme 5]
\item[Theme 6]
\item[Monitoring]
\end{description}
\begin{enumerate}
\item 
\end{enumerate}
\begin{quote}
\end{quote}
